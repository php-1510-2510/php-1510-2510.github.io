\documentclass[]{article}
\usepackage{lmodern}
\usepackage{amssymb,amsmath}
\usepackage{ifxetex,ifluatex}
\usepackage{fixltx2e} % provides \textsubscript
\ifnum 0\ifxetex 1\fi\ifluatex 1\fi=0 % if pdftex
  \usepackage[T1]{fontenc}
  \usepackage[utf8]{inputenc}
\else % if luatex or xelatex
  \ifxetex
    \usepackage{mathspec}
  \else
    \usepackage{fontspec}
  \fi
  \defaultfontfeatures{Ligatures=TeX,Scale=MatchLowercase}
\fi
% use upquote if available, for straight quotes in verbatim environments
\IfFileExists{upquote.sty}{\usepackage{upquote}}{}
% use microtype if available
\IfFileExists{microtype.sty}{%
\usepackage{microtype}
\UseMicrotypeSet[protrusion]{basicmath} % disable protrusion for tt fonts
}{}
\usepackage[margin=1in]{geometry}
\usepackage{hyperref}
\hypersetup{unicode=true,
            pdftitle={Homework 2},
            pdfauthor={Your Name},
            pdfborder={0 0 0},
            breaklinks=true}
\urlstyle{same}  % don't use monospace font for urls
\usepackage{graphicx,grffile}
\makeatletter
\def\maxwidth{\ifdim\Gin@nat@width>\linewidth\linewidth\else\Gin@nat@width\fi}
\def\maxheight{\ifdim\Gin@nat@height>\textheight\textheight\else\Gin@nat@height\fi}
\makeatother
% Scale images if necessary, so that they will not overflow the page
% margins by default, and it is still possible to overwrite the defaults
% using explicit options in \includegraphics[width, height, ...]{}
\setkeys{Gin}{width=\maxwidth,height=\maxheight,keepaspectratio}
\IfFileExists{parskip.sty}{%
\usepackage{parskip}
}{% else
\setlength{\parindent}{0pt}
\setlength{\parskip}{6pt plus 2pt minus 1pt}
}
\setlength{\emergencystretch}{3em}  % prevent overfull lines
\providecommand{\tightlist}{%
  \setlength{\itemsep}{0pt}\setlength{\parskip}{0pt}}
\setcounter{secnumdepth}{0}
% Redefines (sub)paragraphs to behave more like sections
\ifx\paragraph\undefined\else
\let\oldparagraph\paragraph
\renewcommand{\paragraph}[1]{\oldparagraph{#1}\mbox{}}
\fi
\ifx\subparagraph\undefined\else
\let\oldsubparagraph\subparagraph
\renewcommand{\subparagraph}[1]{\oldsubparagraph{#1}\mbox{}}
\fi

%%% Use protect on footnotes to avoid problems with footnotes in titles
\let\rmarkdownfootnote\footnote%
\def\footnote{\protect\rmarkdownfootnote}

%%% Change title format to be more compact
\usepackage{titling}

% Create subtitle command for use in maketitle
\providecommand{\subtitle}[1]{
  \posttitle{
    \begin{center}\large#1\end{center}
    }
}

\setlength{\droptitle}{-2em}

  \title{Homework 2}
    \pretitle{\vspace{\droptitle}\centering\huge}
  \posttitle{\par}
    \author{Your Name}
    \preauthor{\centering\large\emph}
  \postauthor{\par}
      \predate{\centering\large\emph}
  \postdate{\par}
    \date{Due: October 12, 2019 at 11:59pm}


\begin{document}
\maketitle

\subsubsection{Homework Policies:}\label{homework-policies}

\emph{You are encouraged to discuss problem sets with your fellow
students (and with the Course Instructor of course), but you must write
your own final answers, in your own words. Solutions prepared ``in
committee'' or by copying someone else's paper are not acceptable. This
violates the Brown standards of plagiarism, and you will not have the
benefit of having thought about and worked the problem when you take the
examinations.}

\emph{All answers must be in complete sentences and all graphs must be
properly labeled.}

\textbf{\emph{In this homework you will be required to use .Rmd to do
it., you can then knit to a word document of PDF to turn it in.}}

\textbf{\emph{For the PDF Version of this assignment:
\href{https://raw.githubusercontent.com/php-1510-2510/php-1510-2510.github.io/master/homework/hw2.pdf}{PDF}}}

\textbf{\emph{For the R Markdown Version of this assignment:
\href{https://raw.githubusercontent.com/php-1510-2510/php-1510-2510.github.io/master/homework/hw2.Rmd}{RMarkdown}}}

\subsubsection{Turning the Homework in:}\label{turning-the-homework-in}

\emph{Please turn the homework in through canvas. You may use a pdf,
html or word doc file to turn the assignment in.}

\href{https://canvas.brown.edu/courses/1078851/assignments/7744738}{PHP
1510 Assignment Link}

\href{https://canvas.brown.edu/courses/1078852/assignments/7744739}{PHP
2510 Assignment Link}

\subsection{The Data}\label{the-data}

We will work with the dataset called
\href{https://github.com/jennybc/gapminder}{gapminder}, this is a
cleaned up version from \href{http://www.gapminder.org/data/}{Gapminder
Data}. Gapminder contains a lot of great data on all of the nations of
the world. We first need to install the gapminder package in R.

\begin{verbatim}
install.packages("gapminder")
\end{verbatim}

\begin{verbatim}
library(dplyr)
library(gapminder)
gapminder
\end{verbatim}

\subsubsection{Problems for Everyone}\label{problems-for-everyone}

Use \textbf{dplyr} functions to address the following questions. For
some you can just use arrange to print your solutions to the top.

\begin{enumerate}
\def\labelenumi{\arabic{enumi}.}
\tightlist
\item
  How many unique countries are represented per continent?
\item
  Which European nation had the lowest GDP per capita in 1997?
\item
  According to the data available, what was the average life expectancy
  across each continent in the 1980s?
\item
  What 5 countries have the highest total GDP over all years combined?
\item
  What countries and years had life expectancies of \emph{at least} 80
  years? \emph{Only output the columns of interest: country, life
  expectancy and year (in that order).}
\item
  Which three countries have had the most consistent population
  estimates (i.e.~lowest standard deviation) across the years of
  available data?
\item
  Follow the steps below to create a plot about life expectancy.

  \begin{enumerate}
  \def\labelenumii{\alph{enumii}.}
  \tightlist
  \item
    Create a plot of life expectancy over time where each country has
    its own line (\texttt{group=country}).\\
  \item
    Add a layer to this plot where you use
    \texttt{geom\_smooth(method="lm")}
  \item
    Add a layer to this plot where you use
    \texttt{geom\_smooth(method="lm")} but it is colored by continent.
  \end{enumerate}
\item
  Interpret the graph you created in 7.

  \begin{enumerate}
  \def\labelenumii{\alph{enumii}.}
  \tightlist
  \item
    What types of patterns are you seeing?
  \item
    Do all countries follow this pattern?
  \item
    etc\ldots{}
  \end{enumerate}
\item
  Create boxplots of life expectancy by continent. Add a layer using
  \texttt{geom\_jitter()} to see how the points fall in these boxplots.
  **Hint: using \texttt{alpha=0.04} inside the jitter will lighten the
  points*
\item
  Interpret the plot you made in 9.
\end{enumerate}

\subsection{PHP 2510 Only}\label{php-2510-only}

\begin{enumerate}
\def\labelenumi{\arabic{enumi}.}
\setcounter{enumi}{10}
\item
  Which combinations of continent (besides Asia) and year have the
  highest average population across all countries? \emph{your output
  should include all results sorted by highest average population}. With
  what you already know, this one may stump you. See
  \href{http://stackoverflow.com/q/27207963/654296}{this Q\&A} for how
  to \texttt{ungroup} before \texttt{arrange}ing.
\item
  Consider the function below

\begin{verbatim}
hourly_delay <- filter(
          summarise(
              group_by(
                  filter(
                    flights, 
                    !is.na(dep_delay)
                    ),
                    month, day, year, hour
                    ),
                    delay=mean(dep_delay),
                    n=n()
                    ),
                    n>10
                    )
\end{verbatim}
\end{enumerate}

\begin{enumerate}
\def\labelenumi{\alph{enumi}.}
\tightlist
\item
  What are some problems with this function?
\item
  How easy is it to follow the logic of this?
\item
  Rewrite this using piping to make it more understandable.
\item
  Does your rewritten command give the same results?
\end{enumerate}


\end{document}
